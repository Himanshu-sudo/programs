\chapter{Performance Evaluation}

We performed the experiments in Amazon EC2 (https://aws.
amazon.com/ec2/) using python 2.7.13.
\subsection {Implementation Details}
The cloud servers have Ubuntu 18.04 long-term support
(LTS) (https://www.ubuntu.com/) operating system. Also, we
use Docker [24] as the container engine in the servers. The
docker version taken for the implementation is 17.03.2-ce.
To perform the docker checkpoint, the docker experimental
feature is enabled and CRIU (https://criu.org/Main Page) is
installed in the servers. Docker uses CRIU [25] to manage the
lifecycle of processes running inside its containers (https://criu.
org/Docker). We set up the shared storage using NFS (https://help.ubuntu.com/lts/serverguide/network-file-system.html) to
make the checkpointed files available in different servers. We
have taken zookeeper application (https://zookeeper.apache.
org/), PHP web application (https://www.php.net/) and redis
application (https://redis.io/) for evaluation. We have analysed
the output of the docker stats command to check the resource
status of a container. We show the values of the parameters
taken in Amazon EC2 experiments in Table I. The topology
considered for these experiments is shown in Figure 3.
\begin{table}[htbp]
\caption{PARAMETERS IN AMAZON EC2 EXPERIMENT}
\begin{center}
\begin{tabularx}{0.7\textwidth} { 
  | >{\raggedright\arraybackslash}X 
  | >{\centering\arraybackslash}X 
  | >{\raggedleft\arraybackslash}X | }
 \hline
 \textbf{Parameter} & \textbf{Value}  \\
 \hline
 Number of Amazon EC2 virtual
machine instances or worker nodes  & 10    \\
\hline
Processor of a worker node & 2.5 GHz Intel Xeon Family \\ 
\hline 
Number of vCPU in a worker node & 1 \\ 
\hline 
Storage of a worker node & 8 GB \\
\hline 
Operating System used & Ubuntu Server 18.04 LTS \\
\hline
Maximum available RAM of the
worker node & 680 MB \\
\hline
Maximum available RAM of the
manager node & 1780 MB \\
\hline
\end{tabularx}
\label{tab1}
\end{center}
\end{table}
\subsection{Competing Heuristics}
We consider Execution Container Placement and Task assignment Algorithm (EPTA) [19] as a baseline. In EPTA,
the microservice controller queries a small local region for
available physical resources. The execution containers are
placed in the server, which has less node cost as well as
less link cost. The microservice controller requests physical
resources and waits for responses from the physical machines
(PM). Once the request is approved, an execution container
is built on the selected PM. If the request is rejected, the
PM is marked as infeasible. When some tasks have not been
successfully assigned, the microservice controller queries PMs
on a larger scale. We have taken first fit decreasing (FFD) as
another baseline. FFD is used in many works in the existing
literature [26], and it is a well-known heuristic for bin packing
problem. In FFD, the containers are sorted in decreasing order
of the size of the required resource. Then, FFD chooses the
first server that is large enough to place the containers.
